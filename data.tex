\newcommand{\cvfirstname}{Valeriia}
\newcommand{\cvlastname}{Shestakova}
\newcommand{\cvposition}{Software Engineer}

\newcommand{\cvhowami}{
with more than 4 year of production experience in Python development.\\
Have an experience in developing asynchronous web-projects on Python.\\
Was working with web frameworks, such as FastAPI, aiohttp, Django and Flask.\\
Have an expertise in relational and NoSQL databases, microservice architecture, ElasticSearch, AWS, Agile, CI/CD processes, git workflow.\\
As a team member worked using Agile methodology, participated in architecture design, onboarding and interview activities.\\
Always looking forward to new learning opportunities and experience.\\
}

\newcommand{\cvtemplateexperience}{Experience}
\newcommand{\cvexperience}{
	\begin{entrylist}
        \entry
		{2022 – present}
		{Senior Software Engineer, EPAM Systems}
		{Montenegro}
		{
			For now I work as a Key Back End developer on the same project for Liberty Global. Apart from development,
			started taking part in interviewing and onboarding newcomers, tasks decomposition and distribution, architecture design activities.\\
			\texttt{FastAPI}\slashsep\texttt{Python3.10}\slashsep\texttt{Tortoise}
		}
        \entry
		{2019 – 2022}
		{Software Engineer, EPAM Systems}
		{Saint Petersburg}
		{
			Before that, I was working as a Back End developer on one of projects for
			Liberty Global - telecommunication company. The product is a web service for load testing video services.
			The web service has microservice architecture with shared database and written on aiohttp.
			During my work I took part at developing new endpoints, processing data with ElasticSearch,
			developing new microservices, adding AWS functionality.
			Development process includes writing unit tests, performing code reviews, CI/CD process, using PEP8 and typing for Python code.
			Also team is working using Agile methodology, Back End team consists of 3 people [All team is about 20 members].\\
			\texttt{aiohttp}\slashsep\texttt{Python3.7}\slashsep\texttt{PostgreSQL}\slashsep\texttt{RabbitMQ}\slashsep\texttt{ElasticSearch}\slashsep\texttt{Kafka}\\
			\texttt{Redis}\slashsep\texttt{AWS}\slashsep\texttt{Docker}\slashsep\texttt{Nomad}\slashsep\texttt{pytest}\slashsep\texttt{GitLab}
		}
		\entry
		{2020 – 2021}
		{Software Engineer, EPAM Systems}
		{Saint Petersburg}
		{
			In parallel, for a few months I was working on a project for big pharmaceutical company. We was working on
			new project from scratch to the first production release. I was responsible for development new endpoints,
			setup web service using AWS lambdas and step functions, performing data processing with ElasticSearch,
			interaction with database, writing unit tests. The team worked using Agile methodology,
			code review process and other quality gates such as tests, pre-commit, mypy checks.
			Back End team consists of 4 members [All team is about 15 members].\\
			\texttt{Flask}\slashsep\texttt{Python3.7}\slashsep\texttt{AWS Lambda}\slashsep\texttt{DynamoDB}\slashsep\texttt{ElasticSearch}\\
			\texttt{Jenkins}\slashsep\texttt{pytest}\slashsep\texttt{Bitbucket}
		}
	\end{entrylist}
}

\newcommand{\cvtemplateeducation}{Education}
\newcommand{\cveducation}{
	\begin{entrylist}
	    \entry
		{2018}
		{Python course, EPAM Systems}
		{Saint Petersburg}
		{
			Generally I studied Python on EPAM courses. During course, all main Python topics have been learnt
			and put into practice. My final project was written on Python, Flask.
			It was a web service for getting some statistics from external API and displaying it in graphs view.\\
			\texttt{Python3}\slashsep\texttt{Flask}
		}
		\entry
		{2014 – 2020}
		{Bachelor/Master's Degrees in Software Engineering, ITMO University}
		{Saint Petersburg}
		{
			I studied Software Engineering in University. My final project was written on Python, Django.
			It was a web service for saving educational results [certificates] based on Blockchain technology.\\
			\texttt{Python3}\slashsep\texttt{Django}\slashsep\texttt{SQLite}
		}
	\end{entrylist}
}

\newcommand{\cvtemplatecontacts}{Contacts}
\newcommand{\cvcontacts}{
	\icontext{MapMarker}{12}{Montenegro}\\
	\icontext{At}{12}{\href{mailto:lerashestakova@yandex.ru}{lerashestakova@yandex.ru}}\\
	\icontext{Github}{12}{\href{https://github.com/lerushe}{@lerushe}}\\
}
\newcommand{\cvtemplatelanguages}{Languages}
\newcommand{\cvlanguages}{
	\bubbles{
    	5/Russian,
    	4/English
	}{}
}
